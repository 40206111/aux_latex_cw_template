%FILL THESE IN
\def\mytitle{Reflective Report}
\def\myauthor{Emma Parsley}
\def\contact{40206111@live.napier.ac.uk}
\def\mymodule{Group Project (SET09109)}
%YOU DON'T NEED TO TOUCH ANYTHING BELOW
\documentclass[10pt, a4paper]{article}
\usepackage[a4paper,outer=1.5cm,inner=1.5cm,top=1.75cm,bottom=1.5cm]{geometry}
\twocolumn
\usepackage{graphicx}
\graphicspath{{./images/}}
%colour our links, remove weird boxes
\usepackage[colorlinks,linkcolor={black},citecolor={blue!80!black},urlcolor={blue!80!black}]{hyperref}
%Stop indentation on new paragraphs
\usepackage[parfill]{parskip}
%% all this is for Arial
\usepackage[english]{babel}
\usepackage[T1]{fontenc}
\usepackage{uarial}
\renewcommand{\familydefault}{\sfdefault}
%Napier logo top right
\usepackage{watermark}
%Lorem Ipusm dolor please don't leave any in you final repot ;)
\usepackage{lipsum}
\usepackage{xcolor}
\usepackage{listings}
%give us the Capital H that we all know and love
\usepackage{float}
%tone down the linespacing after section titles
\usepackage{titlesec}
%Cool maths printing
\usepackage{amsmath}
%PseudoCode
\usepackage{algorithm2e}

\titlespacing{\subsection}{0pt}{\parskip}{-3pt}
\titlespacing{\subsubsection}{0pt}{\parskip}{-\parskip}
\titlespacing{\paragraph}{0pt}{\parskip}{\parskip}
\newcommand{\figuremacro}[5]{
    \begin{figure}[#1]
        \centering
        \includegraphics[width=#5\columnwidth]{#2}
        \caption[#3]{\textbf{#3}#4}
        \label{fig:#2}
    \end{figure}
}

\lstset{
	escapeinside={/*@}{@*/}, language=C++,
	basicstyle=\fontsize{8.5}{12}\selectfont,
	numbers=left,numbersep=2pt,xleftmargin=2pt,frame=tb,
    columns=fullflexible,showstringspaces=false,tabsize=4,
    keepspaces=true,showtabs=false,showspaces=false,
    backgroundcolor=\color{white}, morekeywords={inline,public,
    class,private,protected,struct},captionpos=t,lineskip=-0.4em,
	aboveskip=10pt, extendedchars=true, breaklines=true,
	prebreak = \raisebox{0ex}[0ex][0ex]{\ensuremath{\hookleftarrow}},
	keywordstyle=\color[rgb]{0,0,1},
	commentstyle=\color[rgb]{0.133,0.545,0.133},
	stringstyle=\color[rgb]{0.627,0.126,0.941}
}

\thiswatermark{\centering \put(336.5,-38.0){\includegraphics[scale=0.8]{logo}} }
\title{\mytitle}
\author{\myauthor\hspace{1em}\\\contact\\Edinburgh Napier University\hspace{0.5em}-\hspace{0.5em}\mymodule}
\date{}
\sloppy
\begin{document}
	\maketitle
    %START FROM HERE
	\section{Introduction}
	
	The aim for this project was to create a 3D marine environment in VR that is procedurally generated. I believe the finished project we produced was well done and surprisingly good looking considering we didn't have a 3D artist. I believe our team worked well together with good communication, verbally and on-line (see Appendix 7.3 Evidence Screen-shots).
	
	\section{Teamwork}
	\subsection{The Team}
	
	We generally worked very well as a team, we helped each other out when we had problems (see Appendix 7.2 Team Reports Week beginning 26th February). The team communicated well letting others know when there were problems or about any information that they missed (see Appendix 7.4 Evidence Screen-shots).
	
	Tim was extremely good at helping everyone out. He helped me with modelling the test terrain and with getting the objects to generate around the player. He helped Drew a lot with problems he was having in unity. He also helped Dillon attempt to put the music into the unity project although that was unsuccessful.
	
	\subsection{My Performance}
	
	I believe I performed well in this group, I made sure to complete a Team report for every week of work for the project manager so they knew how things were progressing(see 7.2 Team Reports). I went to almost every client meeting and group meeting to make sure I was up to date on progress. I also made sure to notify my team-mates if I was expecting to not make a meeting (see Appendix 7.4 Evidence Screen-shots, Missed Meeting).
	
	I also spent some time helping Drew with his collision detection as he couldn't get it to work.
	
	\section{Project Management}
	\subsection{The Team}
	The team was managed well, everyone was required to write and submit weekly reports for our project manager (see Appendix 7.2 Team Reports), this helped them keep an eye on progress and see what the split of the work was so they could accurately fill out the team contribution spreadsheet.\\
	
	Our project manager kept everyone up to date on information, any received grades and any changed to meetings (See Appendix 7.4 Evidence Screen-shots, information relaying).\\
	
	Other team mates also made sure that everyone was kept up to date on information if for whatever reason the project manager did not know it or missed anything (see Appendix 7.4 Evidence Screen-shots, Client information).
	
	\subsection{My Performance}
	
	In my opinion I have done a good job of managing my time in this project, making sure I get something done every week (see Appendix 7.2 Team Reports) and having something to show to my team mates at meetings.\\
	
	I made sure to keep everyone up to date with any information I had and include them in decisions I had to make (see Appendix 7.4 Evidence Screen-shots, 2D artist).\\
	
	When a member was unable to do their job in the group I was able to pick up their task and do it to the best of my ability (see Appendix 7.1 Project Diary, Week Beginning 12th February 2018).
	
	\section{Drive for Results}
	\subsection{The Team}
	
	The project manager kept us up to date on everything. He also wrote all the reports.
	
	Tim worked hard to make a convincing movement system in VR and procedurally animated seaweed.
	
	Drew put in as much effort as he could despite not being able to work with us for a period of time, when he came back he really put the effort in and got everything he was assigned done.
	
	Dillon created many different sound effects for us, he tried very hard to make them sound convincing, despite not being able to actually record sounds underwater. I would've been good if he was in uni more so we could've spent the time with him to help him put the sounds into unity himself, using the plug-in our supervisor recommended.
	
	
	\subsection{My Performance}
	
	I worked hard to make the project something I am proud of and believe you can see obvious improvement of my work throughout the project (see Appendix 7.3 Evidence Screen-shots, Early stages and End state).\\
		
	\section{Progress on Personal Learning Goals}
	
	I have made good progress on my personal learning goals. I did not get to work with the Vive as much as I would've liked to develop my skills in VR as many teams were using the Vive so we couldn't just use it whenever we wanted.\\
	
	I gained a lot of experience as part of a software team and got an idea what it was like to be project managed. It's a shame Drew wasn't available for the whole project as this meant I was working with only one other programmer, so for the programming side of things we were essentially working in a pair rather than a team.
	
	I am now a lot more skilled in using unity in 3D, I hadn't really used this software for many 3D programs before, but as I was working on the procedural generation in this one I worked a lot with this 3D engine throughout the whole project.
	
	\section{Conclusion}
	Despite a few hiccups everything went pretty smoothly and I am very happy with the result we ended up with.
    
    \section{Appendices}
    \subsection{Project Diary}
    \textbf{\textit{Week Beginning 8th January 2018:}}\\
    We started with 3 people in our group which wasn't quite enough to apply for the project we wanted. Before we managed to find our forth member the 3D artist we did have pulled out to do a different project with their friends and so we were back to 2 people both of whom are programmers.\\
    
    Managed to get together a group of 4 people allowing us to apply for the Visualisations of synthetic 3D marine environments project.\\
    
    We were accepted to work on our chosen project.\\
    
    As a group currently consisting of 3 programmers and one project manager we are really hoping to find a 3D artist to create the models for the marine environment.\\
    
    Week Beginning 15th January 2018:\\
    We had our first group meeting and devised a list of questions we wanted to ask our client upon our first meeting.\\
    
    As our client was one of the potential supervisors from the university, at the supervisor meet we asked them who they thought would be the best supervisor for us and they said that they would like to supervise us themselves to avoid any confusion.\\
    
    I received an email from an artist about an artist interested in joining our group. Sadly from further emails it was discovered they were a 2D rather than a 3D artist and we couldn't see how their efforts would be beneficial so we let them down.\\
    
    \textbf{\textit{Week Beginning 22nd January 2018:}}\\
    It was our understanding from the previous week that we were meeting our client at 3 on the monday of this week, however they did not show meaning we did not manage to get specifics on what was wanted on the project as yet.\\
    
    We had our first client meeting this Thursday, they explained to us on how we were to go ahead with the project without a 3D artist. The client also informed us that they were hoping to get pictures and information from the St Abs diving centre to help us with this project. and afterwards thought about how we wanted to split up the team to best complete the work. I was given the task of procedural animation.\\
    
    We moved our communications from Facebook to Slack to better organise the information.\\
    
    A sound design student was added to our team this week, we haven't yet met him but we are hopeful his sound effects will add to the immersion of the final project.\\ 
    
    \textbf{\textit{Week Beginning 29th January 2018:}}\\
    
    We had our fist meeting with our new team member, we discussed his role in the team and how it will work with the project.\\
    
    The PID is due this week so we helped the PM make sure he had all the details he needed and wrote learning goals to be added to the end.\\
    
    The client informed us he was ill this week and thus would not be able to meet with us. This caused a problem for us as the PID needed to be signed and now we would not be meeting the client. The PM looked into solving this by emailing the client (who is also our supervisor) about what to do about this.\\
    
    The PID was submitted on time.\\
    
    I set up a trello for the group and we all added cards to it that showed what we were planning on working on.\\
    
    \textbf{\textit{Week Beginning 5th February 2018:}}
    Now that the PID had been submitted we were ready to begin work by researching our given topics.\\
    
    The client is still ill this week which means they cannot meet with us again which is unfortunate as the meetings are helpful to make sure we are on the right track.\\
    
    \textbf{\textit{Week Beginning 12th February 2018:}}\\
    The client was no longer ill so we got to to have our client meeting this week and we were told we were running a little behind and we need to pick up the pace.\\
    
    This week I created a basic 3D model of the terrain so that procedural generation could be started. The client said they would eventually provide us with a better 3D model of the terrain for us to work with. From talking I discovered all of us in the team had different ideas of what the terrain looks like and so I worked with one of my team members to make sure the temporary terrain fulfilled all of our needs.\\
    
   	I was made aware this week that Drew was unavailable to do work for the foreseeable future. His job, procedural generation, was an essential part of the clients brief so it was reallocated to me.\\
   	
    \textbf{\textit{Week Beginning 19th February 2018:}}
    
    This week I began looking into procedural generation, starting with simply getting cubes to randomly spawn on the previously created terrain. Currently have the generation mapped to a key rather than generating around the player.
    
    \textbf{\textit{Week Beginning 26th February 2018:}}
    This week there was no client meeting due to heavy snow closing the university. This was a shame as the feedback from the client is very helpful, However it did not impede my work as I was able to work fine from home.\\
    
    I got the objects to generate around the player rather than just in the predefined spot. I replaced the simple shapes I had been using with a simple coral model Tim had created.\\
   	
    \textbf{\textit{Week Beginning 5th March 2018:}}\\
    
    This week I added restrictions to generation so objects can only spawn below a certain depth and at certain angles. I then added seaweed to the objects generated and now the environment is really beginning to take shape.\\
    
    Everything seems to be running smoothly within our team despite the fact Drew has been unable to help us. Our project manager finished and submitted the work in progress this week, they seems to be happy with how it came out.
    
    \textbf{\textit{Week Beginning 12th March 2018:}}\\
    Drew is back this week and happy to work. Obviously as I have taken over the procedural generation role I will continue to work on this and a different task was put upon Drew. We discussed it with our client/supervisor and decided Drew would work on an interactive educational system where interacting with an object will tell you something about the object. Unfortunately we still have no information or pictures of the wildlife in the St Abs diving centre so the information will have to be vague based on the crude models we currently have.\\
    
    I really focused on the aesthetics of the project this week so I added fog to decrease visibility, added to the realism and also to stop the spawning and de-spawning of objects in the distance from being as obvious. I also added a variety of different sized and coloured coral and seaweed.
    
    \textbf{\textit{Week Beginning 19th March 2018:}}\\
    I added rarities to the objects so they didn't all spawn at the same rate, however in doing this it became clear that the spread of the objects was unnatural so I worked on getting objects to spawn in clusters as they would in real life to make it seem more natural.
    
    In the client meeting this week they suggested adding slight variations to the colour and size of the objects to make it more natural and visually interesting.
    \textbf{\textit{Week Beginning 26th March 2018 to Week Beginning 2nd April 2018:}}\\
    Little work was done in these weeks as it was the holidays.\\
    
    We were hoping our sound designer could add his sounds into the project this week using a plug-in for Unity. However he had lost his laptop and informed us the plug-in did not work on his computer. As he lives in Glasgow we weren't comfortable asking him to come all the way into the uni in the holidays.\\
    
    \textbf{\textit{Week Beginning 9th April 2018:}}\\
   	This week I added the colour and size variations the client asked for. I also attempted to stop the objects from being able to spawn inside each other however I had issues with this and was not able to get it working.\\
   	
   	The client informed us of the final things they wanted and let us know that we needed to focus on finishing the programming and if sound is to be in the program it will need to be added by the sound designer. However the sound designer was not in this meeting so we weren't able to help them get started on implementing it yet.\\
   	
    \textbf{\textit{Week Beginning 16th April 2018:}}\\
    This week we were making sure everything was ready for the presentation next week.\\
    
    We had the practice presentation were we practised the brief summary of project aims and approach, that was written by our project manager, and we got feedback on that. However we did not have the time in our slot to set the Vive up on the projector, especially as we didn't know how this would work before hand so we were unable to practice our demo.\\
    
    The intention was to get sounds in this week but I understand there were problems with getting the equipment to work and the amount of time the sound designer had to work on the problem.\\
    
    I added boundries to the user this week so that the project was a more fleshed out experience as I made it impossible to get outside the terrain or see where the terrain ends.\\
    
    \textbf{\textit{Week Beginning 23rd April 2018:}}\\
    
    This week we merged everything together so it was all ready and working for our presentation on Thursday.\\
    
    Drew put some of the sounds into the project as he finished other work that was needed.\\
    
    We came in an hour early on the day of the presentation to set up the Vive with the projector. The cable we had was broken and was darked and streaked so we were really unable to use it. Luckily for us Drew ran around the uni looking for another cable and found one just before the presentations started. It took us the full hour to get set up but I think it was worth it as having the project running live on the projector really helped our presentation.\\

	\subsection{Team Reports}
	\textbf{Week Beginning 12th February 2018}\\
	Working on initial terrain for scene so that procedural generation can be started. Created 2 terrains
	one very simple one for easy testing of procedural generation and one more intricate one.\\
	\textbf{Problem}\\
	From the meeting before working on this it became clear that different members of our team had
	very different ideas of what the terrain was to be like. This made me confused and unsure what to
	do.\\
	\textbf{Solution}\\
	I worked with Tim to combine our visions into a piece of terrain that made sense to both of us.\\
	\textbf{Problem}\\
	From this original design I was unsure how to subdivide it to make it look more detailed.\\
	\textbf{Solution}\\
	Tim stepped in and had a go at editing the terrain. What he created seemed reasonable to both of us
	and we kept it as the second more detailed terrain.\\
	\textbf{Next Steps}\\
	Next week I will be starting work on procedural generation as problems have come up for Drew and
	we need to reallocate his responsibilities.\\\\
	
	\textbf{Week Beginning 19th February 2018}\\
	Pulled the repository from GitHub and worked on getting it running on my computer. Looked into
	procedural generation of objects in unity. Started coding some simple procedural generation, started
	with cubes just being generated in random destinations in a predefined area. Continued to work on
	the generation creating cubes only on defined surfaces, tested on all terrains we have created to use
	at the moment and seems to work well. Finally I attempted orientate the objects to the surface they
	spawn on, which from what I can tell also appears to work.\\
	\textbf{Problems}\\
	Was unsure how to go about finding the terrain surface to spawn objects on it. Generated objects
	were half embedded in the terrain. It was hard to see different angles on cubes. Needed to work out
	how an object needs to be rotated to be on the surface. Angles in Unity work in quaternion's, so
	they needed to be converted into that format to rotate objects.\\
	\textbf{Fixes}\\
	Did a raycast from a random point at the max height downward looking for the mesh collider on the
	terrain. Added an offset to stop objects being embedded in the terrain that works temporarily but is
	a hard coded fix and won't work for all objects. Changed objects from cubes to cylinders so it was
	easier to see the angles the object was at. Took the normal of the hit point from the raycast and
	found the angle between that and a vector going straight up, which gave the angle in radians, I then
	converted the angle into degrees multiplied it by an axis I worked out using the same vectors as
	before and then got Unity c\# to convert that into a quaternion rotation.\\
	\textbf{Future Plans}\\
	Look into stopping objects from being able to overlap. Find a better fix for objects being inside
	terrain. Add different objects with different chances of being spawned. Add objects that spawn at
	different depths.\\\\
	
	\textbf{Week Beginning 26th February 2018}\\
	This week I worked on having coral generate around the player, so the player can move in any direction and still have the world procedurally generate around them. I also replaced the cube mesh I was currently using as coral with Tim's coral model.\\
	\textbf{Problem}\\
	When replacing the coral, I noticed the offset for the object was incorrect and the model actually needed to be moved into the ground rather than away from it. This made it clear that we would need to work out the most aesthetically pleasing offset for each model we wish to use rather than being able to work out the offset in the code.\\
	\textbf{Solution}\\
	I added a Coral.cs script to attach to coral prefabs that will contain spawning variables so that the coral spawns in the right area. Currently it is only used for the variable "offset" which tells the procedural script how much to move the unscaled model by to have it correctly positioned on the ground.\\
	\textbf{Problem}\\
	There needs to be some way for more procedural prefabs to be created and managed so that they generate when the player moves towards them\\
	\textbf{Solution}\\
	I was with Tim when working on this so after telling him my ideas for how to get this to work he came up with the idea for a procedural management class to create the procedural prefabs as the player moves around. His system used a method to simplify the player coordinates into a 2D grid based on the size of the procedural area\\
	\textbf{Problem}\\
	The "chunks" of procedural generation should be disabled when the player moves away from them and re-enabled when they move close again.\\
	\textbf{Solution}\\
	Methods were made within the procedural class to create objects if they don't already exist or to enable them if they do, and a Method to disable all objects in that chunk. The procedural manager now looks at the 9 chunks surrounding the player and creates procedural objects where there aren't any already. If the player moves to a different chunk the 9 chunks now surrounding the player are checked against the previous 9 chunks and they are activated and deactivated appropriately\\
	\textbf{Next Steps}\\
	Next week I plan to implement some spawning constraints to the generation so that objects can only be spawned below a certain depth etc.\\\\
	
	\textbf{Week Beginning 5th March 2018}\\
	This week I added restrictions to the coral so that you can set the coral to only spawn at specific
	depths and angles. I made it so multiple different types of coral can be spawned and added Seaweed
	to the list.\\
	\textbf{Problem}\\
	Seaweed doesn't follow the direction of the terrain like coral does, so just adding it to the spawn list
	caused it to look like an oversized display of the terrains normal. The Seaweed shouldn't change its
	angle with the terrain but should rather spawn upright.\\
	\textbf{Solution}\\
	The solution for this was simple as before I changed it this was how the corals spawned by default. I
	simply added a boolean to the coral script called ChangeAngle and then added an if statement to the
	generation that bypasses the process of figuring out the angle of rotation if the Boolean is not true.\\
	\textbf{Problem}\\
	Now that there were restrictions on the placement of the coral I had to decide what to do if an
	object couldn't spawn on the randomly generated position because it was too low, too high, too
	steep, or too shallow.\\
	\textbf{Solution}\\
	To solve this, I gave the procedural a tries variable and the program will loop tries times to try and
	find an object that fits in that spot, if it doesn't manage to find one, it will simply continue on as it
	would if it hadn't found ground to place on.\\
	\textbf{Next Steps}\\
	The next step is to implement rarities so the same amount of each different object doesn't spawn
	everywhere, and also to try and stop objects from spawning inside each other.\\\\
	
	\textbf{Week Beginning 12th March 2018}\\
	This week I focused on the aesthetics of the project. I added fog so that you couldn't see objects
	spawning in in the distance as well. I also added some different sized corals with different colours
	and also some different sized seaweed.\\
	\textbf{Problem}\\
	The user could see objects spawn in the distance and the visibility was completely unrealistic for
	underwater.\\
	\textbf{Solution}\\
	Added fog. I made the fog a blue shade so it emulated the misty blue look of being underwater. This
	decreased the user's visibility making it more realistic and also stopping them from being able to so
	easily see the objects spawn in.\\
	\textbf{Problem}\\
	The skybox was not affected by the fog so coming to the edges of the terrain was strange as you
	could suddenly see the unity default skybox very clearly.\\
	\textbf{Solution}\\
	Changed the skybox to be the same colour as the fog so it is indistinguishable from the fox until you
	get closer to the edge of the terrain than the player should be allowed to go.\\
	\textbf{Problem}\\
	The generated world looked quite boring as there were only really a couple of types of objects in it.\\
	\textbf{Solution}\\
	Added more objects of varying sizes and colour.
	The next steps for me will be to add rarities to the generation so that everything doesn't spawn at
	the same rate.\\\\
	
	\textbf{Week Beginning 19th March 2018}\\
	This week I added rarities to the generation so objects don't all spawn at the same rate.\\
	\textbf{Problem}\\
	Keeping the same sort of spread depending how many objects there are.
	Solution
	The approximate amount of objects in a chunk should be the corals rarity multiplied by the amount
	of objects over 100. This should increase and decrease the amount spawned with the amount of
	objects spawned in a chunk.\\
	\textbf{Problem}\\
	The objects looked really separated, in real life more similar flora and fauna tend to appear together.\\
	\textbf{Solution}\\
	Added clustering to fix this problem, any object spawned can spawn with friends of the same type,
	anywhere from its minimum group size to its maximum group size. This caused the objects to
	generate in groups which made it look a lot nicer.\\
	Next steps is to add slight variations in size and colour to add more visual interest to the scene.\\\\
	
	\textbf{Week Beginning 9th April 2018}\\
	This week and the previous holiday weeks I worked on added colour and size variations in spawning
	as well as stopping objects from being able to spawn inside one another.
	Adding size and colour variations was pretty simple, I only had to look up the specific syntax for
	changing colour and add a random value from a range too that to give everything a slight variation.
	The same was done to randomize the size but I already knew the syntax for that so it was even
	easier.\\
	\textbf{Problem}\\
	Objects can spawn inside each other which is strange when walking around the world.\\
	\textbf{Solution}\\
	Check if an object is colliding with another object and if it is try to move it somewhere in a close
	range, if it fails to move enough times abandon making that object.\\
	\textbf{Problem}\\
	Collider collisions only checked in update but objects are spawned before update is called so it's
	difficult to tell if an object is colliding with another one.\\
	\textbf{Solution}\\
	I have been unable to come up with a solution to this problem as yet.\\
	Next steps are to make sure project is ready to be demoed. To help get everything merged together
	so it is ready to be showed off in full at the presentation. If I have time I will also look into the issue
	with objects spawning inside each other and try to rectify that.\\\\
	
	\textbf{Week Beginning 16th April 2018}\\
	This week I created boundaries for the user. Before this point the user could navigate where ever
	they wanted in the world, be that above the water, through the floor or off the edge of the world.\\
	\textbf{Problem}\\
	The user's hit box needs to be around the player even if the player moves in the physical space they
	have rather than using the movement controls in the virtual space. This requires the hit box to be
	attached to the eye camera's position. However it cannot be simply attached to the eye camera
	component as this would only affect the user's view and not the other parts which were part of the
	user's camera rig.\\
	\textbf{Solution}\\
	The hit box was attached to the parent component of the user's camera rig hierarchy and then
	moved using code to stay around the user's view.
	Problem
	Making the hit box the size of the user standing meant that the user was too big to explore the
	whole environment.\\
	\textbf{Solution}\\
	The hit box was made to be approximately the size of the players head. This allowed them to travel
	right up to the surface of the water so they could reach and put their controller's through the water
	but they cannot see above the water. The user can go down until the ground is about up to their
	chin, which may seem strange but it actually allows for close inspection of the flora and fauna down
	right at the bottom and from usage doesn't seem to cause a loss of emersion.\\
	\textbf{Problem}\\
	When it comes to the borders for the edges of the environment they cannot simply be placed right
	on the edges of the terrain as that would allow users to see that the world does not continue passed
	this point and would ruin emersion.\\
	\textbf{Solution}\\
	The borders had to be carefully placed so the user can travel in as much of the terrain as possible,
	but when they come across a world border they cannot see the end of the terrain. The fog helped
	with this as after a certain distance the world was concealed by the blue sea fog.\\
	\textbf{Problem}\\
	It is very annoying to come across invisible world borders that stop you from moving suddenly. As
	there is no sign they are there, even for me who put them in, it just feels like your controls have
	stopped working rather than that you have reached a border.\\
	\textbf{Solution}\\
	A grid texture was added to the boundaries and a script which disabled and enabled the mesh
	renderer. The mesh renderer would start of disabled, and if it would turn on when the player walked
	into it. This allowed the user to see that there was a border there so they knew why they were being
	stopped. The mesh renderer on the borders originally disabled immediately when the player
	stopped hitting it but this felt strange as you didn't really get enough time to assess the border and
	see where exactly it was blocking. This was rectified by simple giving it a couple seconds of delay
	before it disappeared.\\\\
	
	\textbf{Week Beginning 23rd April 2018}\\
	This was the week of the presentation so I helped get everything merged together and working
	before the presentation.\\
	\textbf{Problem}\\
	The taller seaweed was colliding with the surface of the water causing it's physics based animation
	to get confused and act weirdly.\\
	\textbf{Solution}\\
	There is a way to stop collisions between layers, so the objects where added to appropriate layers
	and the surface water no longer collides with any coral objects and I also stopped the borders from
	colliding with the coral as well.\\
	We managed to merge everything together well and have it all working for our presentation on
	Thursday.
	
	\subsection{Learning Goals}
	I'm looking to gain experience as part of a software team and to get a knowledge of how best to break down a project brief into manageable tasks for individuals to do that will hopefully eventually come together into a good quality product.\\
	
	I am also interested to getting to grips with developing in VR for the HTC Vive, while I haven't coded for it before I am interested in VR and I am excited to teach myself how to work with it.\\
	
	Hopefully by the end of this I will have experience of how to effectively work as group in software, have a better idea of how to use unity in 3D and also have a good understanding of how to develop for VR.
	
    \subsection{Evidence Screen-shots}
	\figuremacro{h}{meet}{Confirmed Meetings}{ - Client confirmed our meeting schedule.}{1.0}
	
	\figuremacro{h}{2d}{2D artist}{ - Discussed with the group what to do about the 2D artist that emailed.}{1.0}
	
	\figuremacro{h}{sa}{Client information}{ - Information relayed to team after meeting client for the fist time.}{1.0}
	
	\figuremacro{h}{original}{Early stages}{ - Original view of procedural generation.}{1.0}
	
	\figuremacro{h}{recent}{End state}{ - End state of procedural generation.}{1.0}
	
	\figuremacro{h}{pm}{Information relaying}{ - PM making sure everyone gets information from client meeting.}{1.0}
	
	\figuremacro{h}{missing}{Missed meeting}{ - Alerting the group that I will not be attending a meeting.}{1.0}
	
		
\end{document}
